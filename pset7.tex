\documentclass{article}
\usepackage[utf8]{inputenc}
\usepackage[shortlabels]{enumitem}
\usepackage{amsmath}
\usepackage{amssymb}
\usepackage[margin=1in]{geometry}
\usepackage{fancyhdr}
\usepackage{listings}
\pagenumbering{gobble}
\begin{document}

\subsubsection*{Practice}
\subsubsection*{From Math 20630}
\begin{center}
\item \subsection*{HW 7: Some Number Theory}
\end{center}

\begin{enumerate}
        \item Use 168.
        \item
            \begin{enumerate}
                \item $n$ is divisible by $p$, so it can be written as $n = ap$ for some $a$. Now
                    since $p$ is prime, if $q$ divides $ap$, then $a$ must be divisible by $q$. Rewriting,
                    we have $n = bqp$ for some $a = bq$, and so $n$ is divisible by $pq$.
                \item Take the starting point of the above argument except $p = q$. We have $n = ap$, and since
                $p = q$, then $n$ is divisible by both $p$ and $q$. However, consider the case when $ap < p^2$. 
                In this case, $n$ is not divisible by $p^2$.
                So $n$ need not be divisible by $pq$ if $p = q$.
            \end{enumerate}
        \item
            \begin{enumerate}
                \item $$x^n - 1 = (x-1)(1+x+x^2+ \cdots + x^{n-1})$$
                    $$=(x + x^2 + \cdots + x^n)+(-1 -x -x^2 - \cdots - x^{n-1})$$
                    Cancelling terms, we get $x^n - 1$ as desired.
                \item If $n$ is not prime, then it can be written as $pa$ where $p$ is some prime number.
                        Then, $2^n - 1$ becomes $2^{pa} - 1$, or $(2^p)^a - 1$. Using the above,
                        $$(2^p)^a - 1 = (2^p-1)(1+2^p+2^{2p}+\cdots + 2^{p(a-1)})$$
                    Notice $2^{p}-1 < 2^n-1$, and $p \leq 2$ in order for $p$ to be prime. As such, $2^p - 1 > 1$,
                    and we have factored our expression into a product of which at least one factor
                    is greater than 1, so $2^n - 1$ is not prime for non-prime $n$.
                \item $n = 11$ is a prime number which does not result in a Mersenne prime. The converse is false.
            \end{enumerate}
        \item 
            \begin{enumerate}
                \item According to the Euclidean algorithm, the gcd of two numbers can be written as a linear combination
                        of those two numbers. The gcd of $p$ and $a$, however, is 1 since $a$ and $p$
                        are coprime. So $\text{gcd}(p, a) = mp + na = 1$
                \item Say $p|a$. If so, we are done. Otherwise, since $p$ does not divide $a$, then $\text{gcd}(p,a) = mp + na = 1$
                        for some $m, n$. Multiplying by $b$, we get $bmp + bna = b$. $bmp$ is obviously divisible by $p$,
                        and $bna$ is divisible by $p$ since we are given that $p|ab$.
            \end{enumerate}
\end{enumerate}

\end{document}